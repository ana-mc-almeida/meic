\documentclass[sigconf]{acmart}


\title{Theme 2: Governance of IT and IT Management}
\subtitle{Introduction of core concepts from the TNO-2023-R11803}

\settopmatter{printacmref=false, authorsperrow=2}
\renewcommand{\footnotetextcopyrightpermission}[1]{} % Remove copyright info

\begin{document}

\settopmatter{printfolios=true}
\pagestyle{plain}

\maketitle

\section{What is the SOC of the Future?}

In recent years, the Security Operations Center (SOC) has become a central component of institutional cybersecurity efforts. Traditionally viewed as a technical and reactive function focused on incident detection and response, the SOC is now undergoing a redefinition. As described in the TNO report “SOC of the Future” (2023), its role is expanding beyond infrastructure protection, becoming a key governance asset embedded in strategic decision-making and risk management. The SOC of the future must respond not only to cyber threats, but also to the complexities of organisational accountability, cross-sector coordination, and the ethical implications of automated security processes.

\section{Current Governance Realities of SOCs}

Despite their growing importance, many SOCs continue to operate with fragmented governance. Lines of accountability are often unclear, especially in organisations where SOC activities are outsourced or spread across multiple departments. Evaluation tools such as SOC-CMM (Cybersecurity Capability Maturity Model) and SIM3 (Security Incident Management Maturity Model) reveal significant maturity gaps, particularly in areas related to strategic integration, governance structures, and institutional alignment.

Rather than being well-integrated within enterprise governance mechanisms, many SOCs function in silos. They rely heavily on technological instrumentation—such as Security Information and Event Management (SIEM) systems—without corresponding institutional mechanisms for oversight, role definition, or escalation. This isolation undermines both the effectiveness and the legitimacy of the SOC as a pillar of digital governance.

\section{Emerging Challenges Between 2024 and 2030}

Looking ahead, the TNO report outlines five major developments that are expected to reshape SOC governance by 2030. First is the convergence of IT, OT (operational technology), and IoT (Internet of Things) environments. This convergence broadens the scope of the SOC and introduces new stakeholders, technical dependencies, and governance risks. Second, the increasing use of artificial intelligence in security operations introduces both efficiency gains and new accountability concerns—particularly when AI systems begin to inform or make operational decisions.

A third trend is the growing prominence of European regulatory frameworks, such as the proposed Cyber Solidarity Act, which aims to enhance cross-border cybersecurity cooperation and establish shared obligations. This implies that SOCs will have to operate within multi-level governance structures, subject to both national mandates and supranational coordination. Fourth, the continued outsourcing of SOC functions to managed service providers (e.g., MSSPs and MDRs) creates structural tension between operational responsiveness and institutional control. Finally, there is a growing demand for transparency, explainability, and strategic contribution from the SOC—transforming it from a reactive function to a strategic actor.

\section{Strategic Implications for IT Governance}

These changes reinforce the need to treat the SOC not merely as a technical service but as a governance function with strategic relevance. The future SOC must be able to justify its decisions, demonstrate compliance with external regulations, and align its operations with broader enterprise objectives. This requires new approaches to metrics, performance evaluation, and integration into governance frameworks such as COBIT or ISO/IEC 27001.

Moreover, as the report suggests, governance cannot stop at internal controls. Organisations increasingly operate in hybrid ecosystems, relying on third-party providers and engaging with public-sector frameworks. This demands a more distributed model of governance—one that is capable of coordinating across organisational boundaries, managing legal and contractual complexity, and maintaining institutional oversight even in federated or outsourced settings.

\section{A Case for Rethinking Governance Structures}

The transformation of the SOC exemplifies broader challenges in the governance of IT and information security. It reflects the tensions between control and agility, between central oversight and distributed execution, and between compliance and innovation. For students engaging with the governance of IT, the SOC offers a rich, practical context to examine how strategic integration is negotiated, how risk is managed across technological and institutional layers, and how governance frameworks must evolve to remain effective.

Ultimately, the SOC of the future becomes a lens through which one can understand the shifting boundaries of responsibility in digital governance. It invites a reconceptualisation of what it means to “govern” IT in a world where threats are continuous, responsibilities are shared, and accountability must be both operational and ethical.

\end{document}
